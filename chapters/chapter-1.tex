\documentclass[../thesis.tex]{subfiles}

\begin{document}
\vspace{-1\baselineskip}

% Background & context
The $20^\text{th}$ century ushered in a revolutionary period for mankind's understanding of the fundamental nature of matter and the forces that govern our universe with the development of special relativity and quantum mechanics, which redefined our understanding of space, time, energy and matter at the furthest extremes of scale from the vast reaches of the cosmos to the tiniest constituents of matter. Building on these principles, Quantum Electrodynamics (QED) \tocite was developed as the first successful quantum field theory (\acs{QFT}) describing electromagnetism. The discovery of beta decay \tocite and its paradoxical behaviors within the framework of \acs{QED} prompted the prediction of neutrinos and development of the theory of weak interaction.

At around the same time, a spectrum of strongly interacting particles was discovered \tocite as particle accelerators probed deeper into atomic nuclei, leading to the formation of the quark model in the 1960s \tocite and with it a hypothesized new binding force, the strong force. However, the \acs{QFT} framework remained incapable of describing the weak and strong interactions until advancements in gauge theory and the incorporation of non-Abelian gauge theory into \acs{QFT} resulted in the formation of Yang-Mills theory \citep{theory:yangmills,theory:yangmills2}. This sparked a renaissance in modern physics with the unification of electromagnetism and weak force in 1967 \tocite under the framework of electroweak (\acs{EW}) theory, as well as the development of Quantum Chromodynamics (\acs{QCD}) \tocite to describe the strong force binding quarks.

At this point, the prediction of massless bosons within \acs{EW} formalism remained a contradiction to the predicted massive $W^\pm$ and $Z$ bosons that mediate the weak force. This was resolved by the introduction of \acs{EW} spontaneous symmetry breaking and the Higgs mechanism in 1964 \citep{theory:higgs1,theory:higgs2,theory:brout_englert}, which explained the generation of masses for both the \acs{EW} bosons and fermions. Together, these developments culminated in the Standard Model of particle physics \acs{SM} \tocite, a comprehensive theory that described the electromagnetic, weak, and strong interactions, classified all known fundamental particles and predicted mathematically consistent but not yet observed particles. Following its inception, particles predicted by the Standard Model were gradually observed experimentally, starting with the gluon in 1979 \tocite, then the $W$ and $Z$ bosons \tocite, and finally the top quark in 1995 \tocite. The final missing piece was confirmed as the Higgs boson was observed in 2012 independently by the \acs{ATLAS} \citep{theory:higgs_atlas} and \acs{CMS} \citep{theory:higgs_cms} detectors at the Large Hadron Collider, completing the Standard Model after a 40-year search and cementing it as the most successful framework so far describing fundamental constituents of matter and their governing forces.

% Problem
Despite its successes, the Standard Model remains incomplete. Key unanswered questions include the nature of dark matter \tocite, which makes up about 27\% of the universe’s energy content but has no explanation within the Standard Model; the origin of neutrino masses and their oscillations \tocite; the observed matter-antimatter asymmetry in the universe \tocite; possible unification of the \acs{EW} and strong interaction into a Grand Unified Theory (\acs{GUT}) \tocite; and the hierarchy problem \tocite describing the large discrepancy in scales between forces and the apparent lightness of the Higgs boson compared to values predicted from quantum corrections. 

% Research question
After the discovery of the Higgs boson, efforts have been underway to construct new hypotheses and models in search of beyond the Standard Model (\acs{BSM}) physics via different avenues, one of which being direct searches at colliders for new resonances or particles not predicted by the \acs{SM}. In particular, the top quark possesses large mass and strong coupling to the Higgs boson \citep{theory:top_coupling} which gives it a special role in many proposed \acs{BSM} models as a possible connection with strong coupling to new particles and heavy resonances. In addition, top quark has a clean decay signature with well-understood final states and is produced in abundance at the \acs{LHC} from \acs{pp} collisions in the form of top pairs \ttbar \tocite. This dissertation presents a search for the production of a heavy resonance that couple preferentially to top quark (top-philic) in association with a top pair (\ttbar) in the final state with either two leptons of the same electric charge or at least three leptons (\acs{SSML}). The search is performed using the four-top (\acs{tttt}) production channel, assuming the hypothesized top-philic heavy resonance decays predominantly to \ttbar.

% Analysis aim & motivation
A similar search for top-philic heavy resonances was performed using a \acs{tttt} final state containing either one lepton or two opposite-sign leptons (\acs{1LOS}) \citep{theory:ttZp_1los} with a much larger branching ratio of 56\%, compared to \acs{SSML} at 12\%. Despite the small cross-section within the \acs{SM}, the \acs{tttt} \acs{SSML} final state provides heightened sensitivity to \acs{BSM} physics and higher signal-to-background ratio than inclusive resonance searches (e.g. in dijet or dilepton final states) due to the distinctive signal signature and suppression of large \acs{SM} background processes in $\ttbar+X$ production \tocite. Additionally, cross-section for \acs{tttt} production can be enhanced by many proposed \acs{BSM} models including see-saw mechanism \tocite, supersymmetry (SUSY) \tocite, Effective Field Theory (EFT) operators \tocite, vector-like quarks \tocite and two-Higgs-doublet models (2HDM) \tocite. Searching within this channel is particularly motivated by the recent observed excess in measurement of four-top production in the \acs{SSML} final state at the \acs{LHC} by the \acs{ATLAS} detector \citep{tttt_obs} with a measured cross-section of $24^{+7}_{-6}$ fb, almost double the \acs{SM} prediction of $13.4^{+1.0}_{-1.8}$ fb.

% Methodology
The search attempts to reconstruct the top-philic resonance directly to search for new physics with minimal dependency on the model choice. In addition, a simplified color-singlet vector boson model \citep{theory:ttZp} is employed for model-dependent interpretations, including setting exclusion limits on the production cross-section and model parameters. Data-driven background estimation methods are implemented for \ttW and charge misidentification background to rectify mismodeling related to jet multiplicity in simulated background that were not covered in the previous \acs{1LOS} search \citep{theory:ttZp_1los}. These methods were employed similarly to great effects in \acs{SM} \acs{tttt} analyses \citep{tttt_evidence,tttt_obs}.

% Thesis structure
This dissertation is organized as follows. Chapter 2 presents the formalism of the \acs{SM} and relevant \acs{BSM} concepts. Chapter 3 provides an introduction to the \acs{LHC} and \acs{ATLAS} detector. Chapter 4 describes the reconstruction and identification of physics object from detector signals. Chapter 5 defines the data and simulated samples used in the analysis. Chapter 6 describes the analysis strategy, including object definition, analysis region description and background estimation methods. Chapter 7 summarizes the uncertainties involved in the analysis. Chapter 8 presents the statistical interpretation and analysis results. Finally, chapter 9 discusses a summary of the analysis and future outlook.



\end{document}