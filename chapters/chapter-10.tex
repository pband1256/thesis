\documentclass[../thesis.tex]{subfiles}

\begin{document}
\vspace{-1\baselineskip}

This dissertation presents a search for \acs{BSM} top-philic heavy vector resonance based on a simplified top-philic color singlet $Z'(\rightarrow\ttbar)$ model in the top-quark pair associated production channel (\ttZp). The search is performed in the same-sign dilepton and multilepton channel of the \acs{tttt} final states, using the full Run 2 data set collected between 2015 and 2018 by the \acs{ATLAS} detector at the \acs{LHC}, corresponding to an integrated luminosity of $140$ \fb of \acs{pp} collisions at center-of-mass energy $\acs{Ecom}=13$ TeV.

New data-driven estimation methods for \ttW and charge misidentification background are employed to improve background modeling and signal sensitivity compared to previous analysis \citep{theory:ttZp_1los}. No significant excess over Standard Model predictions is observed. Observed exclusion limits at 95\% confidence level as a function of the $Z'$ mass are set on the production cross section of $pp\rightarrow\ttZp$ times the $Z'\rightarrow\ttbar$ branching ratio, ranging from 7.9 fb (at $m_{Z'}=2$ TeV) to 9.4 fb (at $m_{Z'}=1$ TeV) depending on the $Z'$ mass. This represent a significant improvement in the exclusion limit for \ttZp \citep{theory:ttZp_1los}, and are currently the most stringent upper limits to date. The analysis probes a $Z'$ mass range from 1 TeV to 3 TeV under the assumption of a top-$Z'$ coupling strength of $c_t=1$ and chirality angle $\theta=\pi/4$.

Further improvements in analysis strategies, including multivariate techniques for signal discrimination, are expected to increase discovery potential in future searches. Looking forward, the upcoming Run 3 data at $\sqrt{s} = 13.6$ TeV will increase the total integrated luminosity by about a factor of 2 \citep{ATL-DAPR-PUB-2023-001} and the $pp\rightarrow\tttt$ cross section by at least 19\% \citep{PhysRevLett.131.211901}, which will help to improve modeling for the \acs{SM} \tttt background. Run 3 improvements along with prospects of the High-Luminosity \acs{LHC} will enhance sensitivity to \acs{BSM} physics and offer more opportunities to explore top-philic resonances and other exciting new phenomena. 

\end{document}