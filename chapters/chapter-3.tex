\documentclass[../thesis.tex]{subfiles}

\begin{document}
\vspace{-1\baselineskip}

\section{The Large Hadron Collider}
\label{sec:LHC}
theoretical predictions are tested with experimental data obtained from particle accelerators\\
world's largest accelerator built by CERN situated on the border of Switzerland and France\\
has been operating since xxxx\\
lifetime divided into 3 runs, currently on Run 3 with planned upgrades on the horizon\\
responsible for a number of discoveries aka Higgs, etc.

\subsection{Overview}
[Basic info: location, size, main working mechanism, main detectors, main physics done]\\
- 27km circumference, reusing LEP tunnels 175m below ground level\\
- 7-13-13.6 TeV center of mass energies for pp collisions\\
- other than pp, also collides pPb, PbPb at 4 points with 4 main detectors: ATLAS, CMS (general purpose detectors), ALICE (heavy ion physics, ion collisions), LHCb ($b$-physics)

\subsection{Operations}
- focuses mainly on pp collisions for this thesis
- beams split into bunches of $1.1\times 10^{11}$ protons with instantaneous luminosity of up to $2\times 10^{34} \text{ cm}^{-2}\text{s}^{-1}$\\
- beam energies ramp up in other accelerators before injection, full ramp up to 6.5 GeV about 20 minutes\\
(insert full diagram of accelerator chain)\\
Linac 4: hydrogen atoms, accelerated up to 160 MeV\\
PSB: H atoms stripped of electrons before injection, accelerated to 2 GeV\\
PS: 26 GeV, SPS: 450 GeV\\
LHC: injection in opposite directions, 6.5 TeV per beam\\


Run 1: 2010-2012, Run 2: 2015-2018, Run 3: 2022-2025, HL-LHC: 2029-?\\
COM energies: 7 \& 8 TeV, 13 TeV, 13.6 TeV, 13.6 \& 14 TeV\\
inbetween periods: long shutdowns (LS1, LS2, LS3)\\
(add HL-LHC timeline graph)

\subsection{Physics}
top factory\\
Higgs studies
(insert SM processes cross sections chart)

\section{The ATLAS detector}
\label{sec:ATLAS}
[goals, coordinate system]\\
right-handed cylindrical system, z-axis follows beamline, azimuthal and polar (0 in the beam direction) angles measured with respect to beam axis.\\
pseudorapidity $\eta = -\ln \tan (\theta/2)$, approaches $\pm\inf$ along and 0 orthogonal to the beamline\\
distance $\Delta R=\sqrt{\Delta \eta^2 + \Delta \phi^2}$\\
transverse momentum \pT component of momentum orthogonal to the beam axis


\subsection{Inner detector}
\subsection{Calorimeter systems}
\subsection{Muon spectrometer}
\subsection{Forward detectors}
\subsection{Magnetic systems}
\subsection{Trigger \& data acquisition}



\end{document}