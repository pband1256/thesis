\documentclass[../thesis.tex]{subfiles}

\begin{document}
\vspace{-1\baselineskip}

Actually, here can just leap right into the tunneling process, basically (no overview)
\begin{itemize}
	\item Introduce types of fission: spontaneous and induced. Can start maybe with historical perspective, b/c I assume they first measured induced
	\item Describe observables for both: the classic timescale idea, and what pops out at different timescales. Most observables are the same for both: lifetimes, fragment yields (primary and cumulative), neutron multiplicities, etc
	\item Describe specific quantities that are different for induced vs spontaneous: cross section, mainly/only
	\item Primarily focus on SF. Dovetail into tunneling process vs outside-of-barrier processes
	\item Talk about tunneling first, even though it'll be longer discussion, b/c that's what happens first
	\item Discussion on post-tunneling process (Langevin, TDDFT, microcanonical)
\end{itemize}

Intro goes here

\section{Overview}
Schmidt2018 is a review

\subsection{Spontaneous Fission}

\subsection{Induced Fission}

\section{The Tunneling Process}

\subsection{Previous Approaches}

\subsection{The Nudged Elastic Band}

\section{Primary Fragment Yields}

\subsection{Dynamical Approaches}
TDDFT, Langevin - Abe1996 is good ref for langevin

\subsection{Microcanonical Ensemble Approach}



\end{document}