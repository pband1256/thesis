\documentclass[../thesis.tex]{subfiles}

\begin{document}
\vspace{-1\baselineskip}

\section{Vertex \& track reconstruction}
Charged particles deposit energy in different layers of the inner detector and muon spectrometer\\

\section{Jets}
\section{Electrons}
[isolation criteria along with muon]
\section{Muons}
\section{Missing transverse momentum}
\section{Topological clustering}
\section{Pile-up \& overlap removal}


\section{$b$-tagging}
- b-jets important object for many analyses (Higgs decay/top quark id)
- special properties of b-jets allowing for identification: decays suppressed by CKM factor, allowing for longer lifetime with a displaced secondary decay vertex, with higher track multiplicity
\subsection{GN2 algorithm}
- GN2 graph neural network based b-tagging algorithm
- brief explanation on how GN2 works
- working points, trade off between efficiency and purity
\subsection*{Calibration}
- correct for b-tagging efficiency disparity between data and MC in the form of a correction scale factor
- ttbar calibration
[find ttbar calibration paper]
- ptrel and high pT calibration
- impact parameter -> signed transverse impact parameter significance
- calibration results




\end{document}