\documentclass[../thesis.tex]{subfiles}

\begin{document}
\vspace{-1\baselineskip}

Events for the analysis first are preselected following a list of criteria to optimize for event quality and background rejection.\\
The criteria are applied sequentially, from top to bottom

\begin{enumerate}
\item \textbf{Good Run List (GRL)}: data events must be part of a predefined list of suitable runs and luminosity blocks.
\item \textbf{Calorimeter cleaning}: events containing signal hits indicating an error in the calorimeter are removed.
\item \textbf{Primary vertex}: events must have at least one reconstructed vertex matched to 2 or more associated tracks with $\pT>500$ MeV.
\item \textbf{Trigger}: events must be selected by at least one trigger documented in \autoref{sec:trigsec}.
\item \textbf{Jet cleaning}: events must pass the LooseBad WP for jet cleaning using jets passing preselection criteria in \autoref{sec:objdef}. This is done to remove events with significant number of calorimeter hits from non-prompt sources (e.g. instrumental effects, cosmic ray background, non-collision particles)
\item \textbf{Bad muon veto}: events are removed if they contain at least one muon before overlap removal with insufficient \pT resolution.
\item \textbf{Kinematic selection}: events must have exactly two Tight leptons with the same electric charge, or at lease three Tight leptons of any charge. The leading lepton must have $\pT>28$ GeV, and all leptons must satisfy $\pT>15$ GeV.
\end{enumerate}

Events are separated into two channels based on the number of leptons: same-sign di-lepton (SS2L) for events with exactly two leptons of the same charge, or multilepton (ML) for events with three or more leptons.\\
Further selections are applied based on the lepton flavors present. In the SS2L channel, if both leptons are electrons, the invariant mass $m_{ll}$ must satisfy $m_{ll}<81$ GeV and $m_{ll}>101$ GeV to suppress background involving $Z$-bosons. In the ML channel, the same criteria must be satisfied for every opposite-sign same-flavor pair of leptons in an event.
\subsection*{Lepton \pT cut}


\section{Object definition}
\label{sec:objdef}

(electrons, muons, jets)\\
Jets are reconstructed using particle-flow method with anti-$k_t$ algorithm, using radius parameter $\Delta R=0.4$.\\
Each selection comes with associated calibration scale factors to account for discrepancies between data and MC simulation, and are applied multiplicatively to the MC event weights.

\begin{table}[!ht]
\centering
\begin{tabular}{l|cccc}
\toprule\toprule
Selection							& Electrons	& Muons	& Jets	\\
\midrule
\midrule
\multirow{ 2}{*}{\pT [GeV]}
	& $>15$ 					& $>15$ 	& $>20$ \\
	& $\pT (l_0)>28$ & & &  \\
\midrule
\multirow{ 2}{*}{$|\eta|$}
	& $1.52\leq|\eta|<2.47$ 	& $<2.5$ 	& $<2.5$ \\
	& $<1.37$ & & &  \\
\midrule
\multirow{ 2}{*}{Identification}
	& \verb|TightLH| 							& \verb|Medium| & NNJvt \verb|FixedEffPt| \\
	& pass \verb|ECIDS| ($ee$/$e\mu$) & 	& ($\pT<60$, $|\eta|<2.4$) \\
\midrule
Isolation
	& \verb|Tight_VarRad| 	& \verb|PflowTight_VarRad| 	& \\
\midrule
Track-vertex assoc. & & & \\
\hspace{3mm} $|d_0^{\mathrm{BL}}(\sigma)|$ 
	& $<5$ 		& $<3$ 		& \\
\hspace{3mm} $|\Delta z_0^{\mathrm{BL}}\sin\theta|$ [mm]
	& $<0.5$ 	& $<0.5$ 	& \\
\end{tabular}
\caption{\label{sel:obj}Caption}%
\end{table}

Jets containing $b$-hadrons are identified and tagged with the \verb|GN2v01| algorithm, described in \autoref{sec:btag}. For this analysis, a jet is considered $b$-tagged if it passes the 85\% WP; this gives the best sensitivity to the signal as shown in \autoref{sec:btagopt}. A \acs{PCBT} score is used to quantify the complete output of the $b$-tagging process for a qualifying jet, with values ranging from 1 to 6 corresponding respectively to the jet not passing any WP, to the jet passing all five WPs.\\

The \acs{ETmiss} reconstruction also applies the NNJvt algorithm to jets, and uses the Tight WP for this analysis.\\

\subsection*{$b$-tagging optimization}
\label{sec:btagopt}
(WP optimization study)


\section{Trigger selection}
\label{sec:trigsec}

\begin{table}[!ht]
\centering
\begin{tabular}{p{9cm}|cccc}
\toprule\toprule
\multicolumn{1}{c|}{\multirow{ 2}{*}{Trigger}}	& \multicolumn{4}{c}{Data period} \\
\multicolumn{1}{c|}{}							& 2015	& 2016	& 2017	& 2018 \\
\midrule
\multicolumn{5}{c}{Single electron triggers} \\
\midrule
\verb|HLT_e24_lhmedium_L1EM20VH| 		& \checkmark & - & - & - \\
\verb|HLT_e60_lhmedium|					& \checkmark & - & - & - \\
\verb|HLT_e120_lhloose|					& \checkmark & - & - & - \\
\verb|HLT_e26_lhtight_nod0_ivarloose|	& - & \checkmark & \checkmark & \checkmark \\
\verb|HLT_e60_lhmedium_nod0|			& - & \checkmark & \checkmark & \checkmark \\
\verb|HLT_e140_lhloose_nod0|			& - & \checkmark & \checkmark & \checkmark \\
\midrule
\multicolumn{5}{c}{Di-electron triggers} \\
\midrule
\verb|HLT_2e12_lhloose_L12EM10VH |		& \checkmark & - & - & - \\
\verb|HLT_2e17_lhvloose_nod0| 			& - & \checkmark & - & - \\
\verb|HLT_2e24_lhvloose_nod0| 			& - & - & \checkmark & \checkmark \\
\verb|HLT_2e17_lhvloose_nod0_L12EM15VHI| 	& - & - & - & \checkmark \\
\midrule
\multicolumn{5}{c}{Single muon trigger} \\
\midrule
\verb|HLT_mu20_iloose_L1MU15| 			& \checkmark & - & - & - \\
\verb|HLT_mu40|							& \checkmark & - & - & - \\
\verb|HLT_mu26_ivarmedium| 				& - & \checkmark & \checkmark & \checkmark \\
\verb|HLT_mu50|							& - & \checkmark & \checkmark & \checkmark \\
\bottomrule\bottomrule
\end{tabular}
\caption{\label{sel:trigger}Caption}%
\end{table}


\end{document}