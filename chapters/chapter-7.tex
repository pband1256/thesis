\documentclass[../thesis.tex]{subfiles}

\begin{document}
\vspace{-1\baselineskip}


\section{Event selection}
Events for the analysis first are preselected following a list of criteria to optimize for event quality and background rejection.\\
The criteria are applied sequentially, from top to bottom

\begin{enumerate}
\item \textbf{Good Run List (GRL)}: data events must be part of a predefined list of suitable runs and luminosity blocks.
\item \textbf{Calorimeter cleaning}: events containing signal hits indicating an error in the calorimeter are removed.
\item \textbf{Primary vertex}: events must have at least one reconstructed vertex matched to 2 or more associated tracks with $\pT>500$ MeV.
\item \textbf{Trigger}: events must be selected by at least one trigger documented in \autoref{sec:trigsec}.
\item \textbf{Jet cleaning}: events must pass the LooseBad WP for jet cleaning using jets passing preselection criteria in \autoref{sec:objdef}. This is done to remove events with significant number of calorimeter hits from non-prompt sources (e.g. instrumental effects, cosmic ray background, non-collision particles)
\item \textbf{Bad muon veto}: events are removed if they contain at least one muon before overlap removal with insufficient \pT resolution.
\item \textbf{Kinematic selection}: events must have exactly two Tight leptons with the same electric charge, or at lease three Tight leptons of any charge. The leading lepton must have $\pT>28$ GeV, and all leptons must satisfy $\pT>15$ GeV.
\end{enumerate}
Events are separated into two channels based on the number of leptons: same-sign di-lepton (SS2L) for events with exactly two leptons of the same charge, or multilepton (ML) for events with three or more leptons. The channels are further separated into regions defined in \autoref{sec:ana} to prepare for analysis.\\
Further selections are applied based on the lepton flavors present. In the SS2L channel, if both leptons are electrons, the invariant mass $m_{ll}$ must satisfy $m_{ll}<81$ GeV and $m_{ll}>101$ GeV to suppress background involving $Z$-bosons. In the ML channel, the same criteria must be satisfied for every opposite-sign same-flavor pair of leptons in an event.

\subsection*{Event categorization}
Simulated events are categorized using truth information of leptons ($e$/$\mu$) and their originating MC particle (mother-particle).\\
Each lepton can be classified as either prompt or non-prompt, with non-prompt leptons further categorized for background estimation purposes.\\
If an event contains only prompt leptons, the event is classified as its corresponding process. If the event contains one non-prompt lepton, the event is classified as the corresponding type of the non-prompt lepton. If the event contains more than one non-prompt lepton, the event is classified as other.
\begin{itemize}
\item \textbf{Prompt}: if the lepton originates from $W$/$Z$/$H$ boson decays, or from a mother-particle created by a final state photon.
\item \textbf{Non-prompt}:
	\begin{itemize}
	\item \textbf{Charge-flip ($e$ only)}: if the reconstructed charge of the lepton differs from that of the first mother-particle.
	\item \textbf{Material conversion ($e$ only)}: if the lepton originated from a photon conversion and the mother-particle is an isolated prompt photon, non-isolated final state photon, or heavy boson.
	\item \textbf{$\gamma$-conversion ($e$ only)}: if the lepton originated from a photon conversion and the mother-particle is a background electron.
	\item \textbf{Heavy flavor decay}: if the lepton originated from a $b$- or $c$-hadron.
	\item \textbf{Fake}: if the lepton originated from a light- or $s$-hadron, or if the truth type of the lepton is hadron.
	\item \textbf{Other}: any lepton that does not belong to one of the above categories.
	\end{itemize}
\end{itemize}

\section{Analysis regions}
\label{sec:ana}
Events are selected and categorized into analysis regions belonging to one of two types: control regions (CRs) enriched in background events, and signal regions (SRs) enriched in signal events. This allows for the examination and control of backgrounds and systematic uncertainties, as well as study of signal sensitivities.\\
The signal is then extracted from the SRs with a profile LH fit using all regions.
The full selection criteria for each region are summarized in \autoref{tab:ana_regions}

\subsection{Signal regions}
- All events selected for SS2L and 3L signal regions must satisfy the following criteria:
\begin{itemize}
\item Contains 6 or more jets, with at least 2 jets $b$-tagged at the 85\% WP
\item Scalar sum of the transverse momenta of all leptons and jets $\HT > 500$ GeV
\item Dilepton invariant mass $m_{\ell\ell}$ does not coincide with the $Z$-boson mass range of $81-101$ GeV
\end{itemize}
- The SR is further granularized by the number of $b$-jets and leptons to further study and improve signal sensitivity\\

\begin{table}[!ht]
\centering
\caption{\label{tab:ana:SR}Caption}%
\begin{tabular}{p{2cm}|p{3cm}l}
\toprule\toprule
\multicolumn{1}{c|}{\multirow{ 2}{*}{SR}} & \multicolumn{2}{c}{Selection criteria} \\
\multicolumn{1}{c|}{}					  & $b$-jets	& leptons \\
\midrule
2b2l			& $N_b = 2$	& $N_l = 2$\\
2b3l4l			& $N_b = 2$	& $N_l \geq 3$\\
3b2l			& $N_b = 3$	& $N_l = 2$\\
3b3l4l			& $N_b = 3$	& $N_l \geq 3$\\
4b				& $N_b = 4$	& \\
\bottomrule\bottomrule
\end{tabular}
\end{table}

\subsection{Control regions}
Control regions are defined for each background to be enriched in the targeted background events, in order to maximize the targeted background's purity and minimize contamination from other sources within the region.\\
This helps to constrain and reduce correlation between background normalization factors.\\
Fit variables and selection criteria are determined via optimization studies on CRs to achieve the largest discriminating power possible between the target background and other event types.
\subsubsection*{\ttW background CRs}
Two types of CRs are defined to estimate the flavor composition and normalization of \ttW +jets background: CR $t\bar{t}W^\pm$+jets to constrain flavor composition, and CR 1b($\pm$) to constrain jet multiplicity spectrum.\\
These are further split into \CRttWpm and \CRonebpm due to the pronounced asymmetry in \ttW production from $pp$ collisions, with \ttWplus being produced at approximately twice the rate of \ttWminus.
Selections on \HT and $N_\mathrm{jets}$ to ensure orthogonality to SR\\
Selections on total charge for each charged $W^\pm$ boson\\

\subsubsection*{Fake/non-prompt background CRs}
Selection for fake/non-prompt CRs are determined using the \verb|DFCommonAddAmbiguity| (DFCAA) variable for reconstructed leptons.

\begin{table}[!ht]
\centering
\caption{\label{tab:ana:DFCAA}Caption}%
\begin{tabular}{p{2cm}|l}
\toprule\toprule
DFCAA & Description \\
\midrule
-1			& No 2nd track found \\
0			& 2nd track found, no conversion found \\
1			& Virtual photon conversion candidate \\
2			& Material conversion candidate\\
\bottomrule\bottomrule
\end{tabular}
\end{table}

Four CRs for three main types of fake/non-prompt backgrounds: virtual photon ($\gamma^{*}$) conversion, photon conversion in detector material (Mat. Conv.) and heavy flavor decays (HF).\\
\begin{itemize}
\item Low $m_\gamma^{*}$: events with an $e^+ e^-$ pair produced from a virtual photon\\
Selects two same-sign leptons with at least one electron reconstructed as an internal conversion candidate and neither as with a material conversion candidate ($\mathrm{DFCAA}_{\ell_1(\ell_2)}=1$ and $\neq 2$)\\
NF constrained using yield count only.
\item Mat. Conv.: events with an electron originating from photon conversion within the detector material.\\
Selects two same-sign leptons with at least one electron reconstructed as a material conversion candidate ($\mathrm{DFCAA}_{\ell_1(\ell_2)}=2$).\\
NF constrained using yield count only.
\item HF $e/\mu$: events with a reconstructed non-prompt lepton from semi-leptonic decays of $b$- and $c$-hadrons (heavy flavor decays)\\
Selects three leptons with at least two electrons/muons, with no lepton reconstructed as a conversion candidate ($\mathrm{DFCAA}<0$).\\
NFs constrained by fitting with \pT of the third leading lepton $\ell_3$.
\end{itemize}
\subsection{Validation regions}
In addition, validation regions are also defined to validate the normalization and modeling of \ttZ and \ttW background without being used in the fit.

\begin{itemize}
\item \ttZ : Selects events with at least two $b$-tagged jets, at least four total jets and three leptons with at least one same-flavor opposite-sign lepton pair possessing invariant mass $m_{\ell\ell}$ within the $Z$-boson mass window of $81-101$ GeV
\item \ttW : Main charge asymmetric background leaning \ttWplus, validated using the difference in number of positively and negatively charged events $N_{+}-N_{-}$ instead of total number of events.\\
Selects using CR \ttW and CR 1b criteria, with one VR not orthogonal to SR and one orthogonal VR with more limited statistics.
\end{itemize}

\begin{table}[!htbp]
\centering
\caption{\label{tab:ana_regions}Caption}%
\resizebox{\textwidth}{!}{%
\begin{tabular}{l|lllll}
\toprule\toprule
Region & Channel & $N_\mathrm{jets}$ & $N_b$ & Other selections & Fitted variable \\
\midrule
\midrule
\multirow{2}{*}{CR Low $m_{\gamma^{*}}$}	& \multirow{2}{*}{SS $e\ell$} 	& \multirow{2}{*}{$[4,6)$}	& \multirow{2}{*}{$\geq 1$}	&
	$\ell_1/\ell_2$ is from virtual photon decay 	& \multirow{2}{*}{event yield} \\
	& & & & $\ell_1+\ell_2$ not from material conversion & \\[6pt]
CR Mat. Conv. 			& SS $e\ell$ 	& $[4,6)$	& $\geq 1$	&
	$\ell_1/\ell_2$ is from material conversion 	& event yield \\[6pt]
\multirow{4}{*}{CR HF $\mu$} 			& \multirow{4}{*}{$\ell\mu\mu$} 	& \multirow{4}{*}{$\geq 1$}	& \multirow{4}{*}{$1$}		&
	$\ell_1+\ell_2$ not conversion candidates & \multirow{4}{*}{$\pT(\ell_3)$} \\
	& & & & $100 < \HT < 300$ GeV & \\
	& & & & $\ETmiss > 35$ GeV & \\
	& & & & total charge $= \pm 1$ & \\[6pt]
\multirow{4}{*}{CR HF $e$} 				& \multirow{4}{*}{$ee\ell$}		& \multirow{4}{*}{$\geq 1$}	& \multirow{4}{*}{$1$}		&
	$\ell_1+\ell_2$ not conversion candidates & \multirow{4}{*}{$\pT(\ell_3)$} \\
	& & & & $100 < \HT < 275$ GeV & \\
	& & & & $\ETmiss > 35$ GeV & \\
	& & & & total charge $= \pm 1$ & \\[6pt]
\multirow{4}{*}{CR \ttWplus}	 		& \multirow{4}{*}{SS $\ell\mu$}	& \multirow{4}{*}{$\geq 4$}	& \multirow{4}{*}{$\geq 2$}	&
	$|\eta(e)|<1.5$ & \multirow{4}{*}{$N_\mathrm{jets}$} \\
	& & & & for $N_b=2$: $\HT < 500$ GeV or $N_\mathrm{jets}<6$ & \\
	& & & & for $N_b\geq 3$: $\HT < 500$ GeV & \\
	& & & & total charge $> 0$ & \\[6pt]
\multirow{4}{*}{CR \ttWminus}			& \multirow{4}{*}{SS $\ell\mu$}	& \multirow{4}{*}{$\geq 4$}	& \multirow{4}{*}{$\geq 2$}	& 
	$|\eta(e)|<1.5$ & \multirow{4}{*}{$N_\mathrm{jets}$} \\
	& & & & for $N_b=2$: $\HT < 500$ GeV or $N_\mathrm{jets}<6$ & \\
	& & & & for $N_b\geq 3$: $\HT < 500$ GeV & \\
	& & & & total charge $< 0$ & \\[6pt]
\multirow{3}{*}{CR 1b(+)} 				& \multirow{3}{*}{SS2L+3L}		& \multirow{3}{*}{$\geq 4$}	& \multirow{3}{*}{$1$}		& 
	$\ell_1+\ell_2$ not from material conversion & \multirow{3}{*}{$N_\mathrm{jets}$} \\
	& & & & $\HT > 500$ GeV & \\
	& & & & total charge $> 0$ & \\[6pt]
\multirow{3}{*}{CR 1b(-)} 				& \multirow{3}{*}{SS2L+3L} 		& \multirow{3}{*}{$\geq 4$}	& \multirow{3}{*}{$1$}		& 
	$\ell_1+\ell_2$ not from material conversion & \multirow{3}{*}{$N_\mathrm{jets}$} \\
	& & & & $\HT > 500$ GeV & \\
	& & & & total charge $< 0$ & \\
\midrule
VR \ttZ 		& 3L $\ell^{\pm}\ell^{\mp}$	& $\geq 4$ 	& $\geq 2$ 	
	& $m_{\ell\ell} \in [81,101]$ GeV	& $N_\mathrm{jets}, m_{\ell\ell}$ \\[6pt]
VR \ttW+1b 	& SS2L+3L 	& 			& 			
	& CR $t\bar{t}W^{\pm}$ $||$ CR 1b($\pm$) 			& $N_\mathrm{jets}$ \\[6pt]
VR \ttW+1b+SR 	& SS2L+3L 	& 			& 			
	& CR $t\bar{t}W^{\pm}$ $||$ CR 1b($\pm$) $||$ SR 	& $N_\mathrm{jets}$ \\[6pt]
\midrule
\midrule
\multirow{2}{*}{SR} 					& \multirow{2}{*}{SS2L+3L} 		& \multirow{2}{*}{$\geq 6$}	& \multirow{2}{*}{$\geq 2$}	& 
	$\HT > 500$ GeV & \multirow{2}{*}{\HT} \\
	& & & & $m_{\ell\ell} \notin [81,101]$ GeV & \\
\bottomrule\bottomrule
\end{tabular}}
\end{table}


\section{Background estimation}
\label{sec:bg}
Background events in this analysis consist of SM processes that can result in a \tttt SSML final state.\\
Can be divided into two types: reducible and irreducible.\\
Reducible background consists of processes that do not result in SSML final state physically, but are reconstructed as such due to erroneous detector and reconstruction effects.\\
Three main types: charge misidentification (QmisID), fake leptons and non-prompt leptons.\\
Estimated using template fitting method to adjust MC predictions via floating normalization factors constrained in the CRs.\\
Irreducible background consists of SM processes that result in SSML final states physically, with all leptons being prompt.\\
Main irreducible background considered in this analysis: \tttt, \ttW, \ttZ, and \ttH with smaller contributions from $VV$, $VVV$, $VH$ and rarer processes like \ttVV, \tWZ, \tZq and \ttt.\\
Most irreducible backgrounds are estimated using MC simulations normalized to their theoretical SM cross sections (template fitting), with the exception of \ttW background due to MC mismodeling of the process at high jet multiplicities.\\
The \ttW is instead given four dedicated CRs, and estimated using a data-driven method with a fitted function parameterized in $N_\mathrm{jets}$\\
All CRs and SR are included in the final LH-fit to data.


%\subsection{Fake \& non-prompt leptons}
%\subsection{Reducible background}
\subsection{Template fitting for fake/non-prompt estimation}
\label{sec:template}
Template fit method is a semi-data-driven approach that estimates fake/non-prompt background distributions by fitting the MC kinematic profiles of background processes arising from fake/non-prompt leptons to data.\\
Each of the four main sources of fake/non-prompt leptons is assigned a free-floating normalization factor constrained by a CR enriched with the corresponding background. The NFs are determined simultaneously with the signal.
\begin{itemize}
\item $\mathrm{NF}_{\text{HF }e(\mu)}$: events with one reconstructed non-prompt electron (muon) from heavy flavor decays,
\item $\mathrm{NF}_{\text{Mat. Conv.}}$: events with one reconstructed non-prompt electrons from photon conversion in the detector material
\item $\mathrm{NF}_{\text{Low }m_{\gamma^{*}}}$: events with one reconstructed non-prompt electrons in an $e^+e^-$ pair from virtual photon ($\gamma^*$) conversion.
\end{itemize}


\subsection{Charge misidentification data-driven estimation}
\label{sec:qmisid}
The same-sign di-lepton channel in the analysis gives rise to a major background contamination in opposite-sign di-lepton events with one misidentified charge.\\
Charge misidentification occurs via incorrect track curvature measurements or trident electron contamination from bremsstrahlung, and therefore mainly concerns electrons due to muons' low bremsstrahlung rate and precise curvature information using the ID and MS.\\
The charge misidentification rates is significant at higher \pT and varies with $|\eta|$ as a proxy for the amount of detector material the electron interacted with, and is consequently estimated in this analysis using a data-driven method with assistance from ECIDS.\\
The charge flip probability $\epsilon$ is estimated using a sample of $Z\rightarrow e^{+}e^{-}$ events with additional constraints on the invariant mass $m_{ee}$ to be within 10 GeV of the $Z$-boson mass.\\
The $Z$-boson mass window is defined to be within $4\sigma$ to include most events within the peak, and is determined by fitting the $m_{ee}$ spectrum of the two leading electrons to a Breit-Wigner function, resulting in a range of $[65.57, 113.49]$ for SS events and $[71.81, 109.89]$ for OS events. Background contamination near the peak is assumed to be uniform and subtracted using a sideband method.\\
Since the $Z$-boson decay products consist of a pair of opposite-sign electrons, all same-sign electron pairs are considered to be affected by charge misidentification.\\
Assuming the charge flip probabilities of electrons in an event are uncorrelated, the number of events with same-sign electrons $N_{ij}^\mathrm{SS}$ with the leading electron in the $i^\mathrm{th}$ 2D bin in $(\pT,|\eta|)$ and the sub-leading electron in the $j^\mathrm{th}$ bin can be estimated as
\begin{equation}
N_{ij}^\mathrm{SS} = N_{ij}^\mathrm{tot} (\epsilon_i(1-\epsilon_j) + \epsilon_j(1-\epsilon_i),
\end{equation}
where $N_{ij}^\mathrm{tot}$ is the total number of events in the $i^\mathrm{th}$ and $j^\mathrm{th}$ bin regardless of charge, and $\epsilon_{i(j)}$ is the charge flip rate in the $i^\mathrm{th}$($j^\mathrm{th}$) bin.\\
Assuming $N_{ij}^\mathrm{SS}$ follows a Poisson distribution around the expectation value $\bar{N}_{ij}^\mathrm{SS}$, the charge flip rate $\epsilon$ can be estimated by minimizing a negative-LLH function parameterized in \pT and $|\eta|$,
\begin{align}
-\ln (\mathcal{L}(\epsilon | N_\mathrm{SS})) 
&= -\ln \mathlarger{\prod}_{ij} \frac{(N_{ij}^\mathrm{tot})^{N_{ij}^\mathrm{SS}} \cdot e^{N_{ij}^\mathrm{tot}}}{N_{ij}^\mathrm{SS}!} \\
&= -\sum_{ij} \left[
N_{ij}^\mathrm{SS} \ln (
N_{ij}^\mathrm{tot}( \epsilon_i(1-\epsilon_j)+\epsilon_j(1-\epsilon_i) )) - 
N_{ij}^\mathrm{tot}( \epsilon_i(1-\epsilon_j)+\epsilon_j(1-\epsilon_i) ) \right].
\end{align}
The charge flip rate is then calculated separately for SR and CRs with different electron definitions (CR Low $m_{\gamma^{*}}$, CR Mat. Conv., CR \ttW) using events satisfying 2LSS kinematic selections but with OS electrons, after applying region-specific lepton selections and ECIDS. The following weight is applied to OS events to correct for misidentified SS events within the region:
\begin{equation}
w = \frac{\epsilon_i+\epsilon_j-2\epsilon_i\epsilon_j}{1-\epsilon_i-\epsilon_j+2\epsilon_i\epsilon_j}.
\end{equation}



\subsection{\ttW background data-driven estimation}
- \ttW represents a major source of irreducible background contamination in SM and BSM analyses with \tttt final states.\\
- Measured cross section for \ttW background has been consistently higher than predicted values as seen in previous analyses (\ttH/\ttW multilepton \citep{bg:ttH_ttW_ML}\cite{ana:ttW_meas}, \tttt analyses \citep{tttt_evidence}\citep{tttt_obs}) due to mismodeling, especially at higher $N_\mathrm{jets}$ \\
(show postfit \ttW VR distribution)\\
- Previously, this was handled by assigning large ad-hoc systematic uncertainties to \ttW events with 7 or more jets. 
% expand on tttt VR discrepancy?
- A semi-data-driven method originally employed in the R-parity-violating-supersymmetry search \citep{ana:r_par_susy_2021} was used to mitigate this problem. 
- This method was shown to be effective in the SM \tttt observation analysis \citep{ana:tttt_obs} by improving \ttW modeling especially in the showering step and switching \ttW systematic uncertainties from predominantly modeling to statistical.\\
- MC kinematic distibutions for \ttW are applied with correction factors obtained from a fitted function parameterized in $N_\mathrm{jets}$.\\
- The function describes scaling patterns for QCD \citep{bg:qcd_scaling} can be represented by ratio of successive exclusive jet cross-sections
\begin{equation}
R_{(n+1)/n} = e^{-b} + \frac{\bar{n}}{n+1} = a_0 + \frac{a_1}{1+(j-4)},
\end{equation}
where $n$ is the number of jets in addition to the hard process, $j$ is the inclusive number of jets, and $\bar{n}$ is the expectation value for the Poisson distribution for exclusive jet cross-section at jet multiplicity $n$, described as $P_n=\sigma_n/\sigma_\mathrm{tot}$.\\
- Same-sign di-lepton \ttW events dominate the \ttW background and produce $4$ jets in the matrix element at tree level for the hard process, so $n$ is defined starting from $5$ jets and $j$ is defined as inclusive number of jets with 4 or more jets, or $j\equiv n+4$.\\
- The two terms in the equation correspond respectively to staircase and Poisson scaling between successive multiplicity cross sections, defined as constant ratios $e^{-b}$ and ratios between Poisson probability for $n+1$ and $n$ jets. Staircase scaling is sensitive to events with high jet multiplicity, while Poisson scaling is sensitive to events with low jet multiplicity \citep{bg:qcd_scaling}.\\
- The scaling pattern can then be re-parameterized in $a_0$ and $a_1$ to obtain the \ttW yield at $j'$
\begin{equation}
\mathrm{Yield}_{\ttW(j')} = \mathrm{Yield}_{\ttW(j=4)} \times 
\mathlarger{\prod}_{j=4}^{j'-1} \left( a_0 + \frac{a_1}{1+(j-4)} \right)
\end{equation}
where $j'$ is defined as $j'\equiv j+1$ with $j\geq 4$ since the parameterization starts at the $4^\mathrm{th}$ jet.\\
The \ttW yield at the 4-jet bin can be represented by a normalization factor applied to \ttW MC simulation as $\mathrm{Yield}_{\ttW(j=4)}=\mathrm{NF}_{\ttW(j=4)} \times \mathrm{MC}_{j=4}$.\\
To account for the disparity in \ttWplus and \ttWminus cross-section, assuming the scaling is the same for both processes, $\mathrm{NF}_{\ttW(j=4)}$ can be further split into $\mathrm{NF}_{\ttWplus(j=4)}$ and $\mathrm{NF}_{\ttWminus(j=4)}$. Both NFs are left free-floating to constrain \ttW yields at the 4-jet bin in \CRonebp and \CRonebm.\\
The final $N_\mathrm{jets}$-parameterized function can then be represented by $\mathrm{NF}_{\ttW(j')}$ as
\begin{equation}
\label{eq:ttWdd}
\mathrm{NF}_{\ttW(j')} = \left(\mathrm{NF}_{\ttWplus(j=4)} + \mathrm{NF}_{\ttWminus(j=4)}\right) \times \mathlarger{\prod}_{j=4}^{j'-1} \left( a_0 + \frac{a_1}{1+(j-4)} \right).
\end{equation}
This normalization is calculated and applied separately for each sub-sample of \ttWplus and \ttWminus in an $N_\mathrm{jets}$ bin for $4\leq N_\mathrm{jets}<10$.\\
Due to small contributions in the CRs, events with $N_\mathrm{jets}<4$ and $N_\mathrm{jets}\geq 10$ are not normalized with this scheme.\\
Instead, $N_\mathrm{jets}<4$ \ttW events are fitted by propagating normalization in the 4-jet bin without additional shape correction. The correction factor for \ttW events with $N_\mathrm{jets}\geq 10$ is obtained by summing up the overflow from $N_\mathrm{jets}=10$ to $N_\mathrm{jets}=12$, described as $\sum_{j'=10}^{12} \prod_{j=4}^{j'-1}\left(a_0+\frac{a_1}{1+(j-4)}\right)$. Events with $N_\mathrm{jets}\geq 13$ are negligible and thus not included in the sum.

\subsubsection*{Control region definitions}
Four control regions \CRttWp, \CRttWm, \CRonebp, \CRonebm are constructed to fit $\mathrm{NF}_{\ttWpm(j=4)}$ and the scaling parameters $a_0$, $a_1$ for the \ttW background, as well as validating the parameterization.\\
Events in \CRttWpm are required to contain at least two $b$-tagged jets similar to the SR to determine the \ttW normalization within an SR-related phase space. Orthogonality with SR is satisfied by requiring $\HT<500$ GeV or $N_\mathrm{jets}<6$ when $N_b=2$, and $\HT<500$ GeV when $N_b\geq 3$.\\
The remaining \CRonebpm require events to have $\HT>500$ GeV and at least four jets to encompass events with high $N_\mathrm{jets}$, which can be used to determine the \ttW jet multiplicity spectrum for fitting $a_{0,1}$. The selection criteria also include exactly one $b$-tagged jet to maintain orthogonality with SR. Assuming the \ttW jet multiplicity distribution is similar across different $N_b$, a fitted $N_\mathrm{jets}$ distribution in \CRonebpm can be used to describe the \ttW parameterization at higher $N_\mathrm{jets}$. The full selection criteria for all four regions are shown in \autoref{tab:ana:regions}\\

Validating the \ttW parameterization in \autoref{eq:ttWdd} makes use of the unique charge asymmetry in \ttW production that's not present in other background or signal processes. The number of events with all negatively charged leptons is subtracted from that of events with all positively charged leptons, which cancels out charge symmetric events and leaves the \ttW background. Validation is done via a statistical-only (stat-only) fit to the \ttW MC prediction in \CRonebpm.


%\subsection{\ttZ background validation}



\end{document}