\documentclass[../thesis.tex]{subfiles}

\begin{document}
\vspace{-1\baselineskip}

\section{Analysis regions}
\label{sec:regions}

\subsection{Event categorization}
Simulated events are categorized using truth information of leptons ($e$/$\mu$) and their originating MC particle (mother-particle).\\
Each lepton can be classified as either prompt or non-prompt, with non-prompt leptons further categorized for background estimation purposes.\\
If an event contains only prompt leptons, the event is classified as its corresponding process. If the event contains one non-prompt lepton, the event is classified as the corresponding type of the non-prompt lepton. If the event contains more than one non-prompt lepton, the event is classified as other.
\begin{itemize}
\item \textbf{Prompt}: if the lepton originates from $W$/$Z$/$H$ boson decays, or from a mother-particle created by a final state photon.
\item \textbf{Non-prompt}:
	\begin{itemize}
	\item \textbf{Charge-flip ($e$ only)}: if the reconstructed charge of the lepton differs from that of the first mother-particle.
	\item \textbf{Material conversion ($e$ only)}: if the lepton originated from a photon conversion and the mother-particle is an isolated prompt photon, non-isolated final state photon, or heavy boson.
	\item \textbf{$\gamma$-conversion ($e$ only)}: if the lepton originated from a photon conversion and the mother-particle is a background electron.
	\item \textbf{Heavy flavor decay}: if the lepton originated from a $b$- or $c$-hadron.
	\item \textbf{Fake}: if the lepton originated from a light- or $s$-hadron, or if the truth type of the lepton is hadron.
	\item \textbf{Other}: any lepton that does not belong to one of the above categories.
	\end{itemize}
\end{itemize}

\subsection{Control regions}
\subsubsection*{$t\bar{t}W$ CRs}
\subsection{Signal regions}
[include blinding strategy]
\subsection{Validation region}


\section{Background estimation}
\label{sec:bg}
\subsection{Fake \& non-prompt leptons}
\subsection{Irreducible background}


\section{Signal extraction}
\label{sec:mva}
%\subsection*{SM MVA}
%\subsection*{BSM MVA}



\end{document}