\chapter{Photon processes}
%\label{sec:GJ}

The following paragraphs describe the set-up of the current ATLAS \(\gamma\)+jets and \(\gamma\gamma\)+jets baseline samples.

\section[Sherpa (MEPS@NLO)]{\SHERPA (\MEPSatNLO)}
%\label{sec:gammajets-sherpa-nlo}

\subsection*{Samples}
%\label{sec:gammajets-sherpa-nlo-samples}

The descriptions below correspond to the samples in \cref{tab:gammajets-sherpa-nlo}.

\begin{table}[!htbp]
  \caption{\(\gamma\)+jets and \(\gamma\gamma\)+jets samples with \SHERPA NLO\@.}%
  \label{tab:gammajets-sherpa-nlo}
  \centering
  \begin{tabular}{l l}
    \toprule
    DSID range & Description \\
    \midrule
    364541--364547 & single photon \\
    364350--364354 & diphoton  \\
    \bottomrule
  \end{tabular}
\end{table}


%same description as in~\cref{sec:vjets-sherpa-vjets}
\subsection{\(\gamma\)+jets}
%\label{sec:gammajets-sherpa-nlo-singlephoton}

\paragraph{Short description:}

Prompt single-photon production was simulated with the
\SHERPA[2.2]~\cite{Bothmann:2019yzt} generator. In this set-up, NLO-accurate
matrix elements for up to two partons, and LO-accurate matrix elements for up
to four partons were calculated with the Comix~\cite{Gleisberg:2008fv} and
\OPENLOOPS~\cite{Buccioni:2019sur,Cascioli:2011va,Denner:2016kdg} libraries. They were matched
with the \SHERPA parton shower~\cite{Schumann:2007mg} using the \MEPSatNLO
prescription~\cite{Hoeche:2011fd,Hoeche:2012yf,Catani:2001cc,Hoeche:2009rj}
with a dynamic merging cut~\cite{Siegert:2016bre} of \qty{20}{\GeV}.
Photons were required to be isolated according to
a smooth-cone isolation criterion~\cite{Frixione:1998jh}. Samples were generated using the
\NNPDF[3.0nnlo] PDF set~\cite{Ball:2014uwa}, along with the dedicated set of
tuned parton-shower parameters developed by the \SHERPA authors.


\paragraph{Long description:}

Prompt single-photon production was simulated with the
\SHERPA[2.2]~\cite{Bothmann:2019yzt} parton shower Monte Carlo
generator. In this set-up, NLO-accurate
matrix elements for up to two partons, and LO-accurate matrix elements for up
to four partons were calculated with the Comix~\cite{Gleisberg:2008fv} and
\OPENLOOPS~\cite{Buccioni:2019sur,Cascioli:2011va,Denner:2016kdg} libraries.
The default \SHERPA parton shower~\cite{Schumann:2007mg} based on
Catani--Seymour dipole factorisation and the cluster hadronisation model~\cite{Winter:2003tt}
were used. They employed the dedicated set of tuned parameters developed by the
\SHERPA authors for this generator version and the \NNPDF[3.0nnlo] PDF
set~\cite{Ball:2014uwa}.

The NLO matrix elements for a given jet multiplicity were matched to the parton
shower using a colour-exact variant of the MC@NLO
algorithm~\cite{Hoeche:2011fd}. Different jet multiplicities were then merged
into an inclusive sample using an improved CKKW matching
procedure~\cite{Catani:2001cc,Hoeche:2009rj} which was extended to NLO
accuracy using the \MEPSatNLO prescription~\cite{Hoeche:2012yf}.
The merging cut was set dynamically at a scale of \qty{20}{\GeV}
according to the prescription in Ref.~\cite{Siegert:2016bre}.

The renormalisation and factorisation scales for the photon-plus-jet core
process were set to the transverse energy of the photon, \(E_\text{T}^\gamma\).
The strong coupling constant was set to \(\alphas (m_Z)= 0.118\) and the QED coupling
constant was evaluated in the Thomson limit. Photons from the matrix elements were required
to be central, by being within the rapidity range \(|y_{\gamma}|<2.7\), and isolated according to a
smooth-cone isolation criterion~\cite{Frixione:1998jh} with \(\delta_0=0.1\), \(\epsilon_{\gamma}=0.1\) and \(n=2\).

The effects of QCD scale uncertainties were evaluated~\cite{Bothmann:2016nao} using
seven-point variations of the factorisation and renormalisation scales in the matrix elements.
The scales were varied independently by factors of \(0.5\) and \(2\), avoiding variations in opposite directions.

PDF uncertainties for the nominal PDF set were
evaluated using the 100 variation replicas, as well as \(\pm 0.001\) shifts
of \alphas. Additionally, the results were cross-checked using the central values of the
\CT[14nnlo]~\cite{Dulat:2015mca} and \MMHT[nnlo]~\cite{Harland-Lang:2014zoa}
PDF sets.

%\textcolor{red}{ Several Monte Carlo slices were generated to ensure good statistics over the whole phase space. In each of them, the photon was required to fall within a \(E_{T}^{\gamma}\) range that depends on the slice avoiding any double-counting when the different slices are combined.}


\subsection[yy+jets]{ \(\gamma\gamma\)+jets}
%\label{sec:gammajets-sherpa-nlo-diphoton}

\paragraph{Short description:}

Prompt diphoton production was simulated with the
\SHERPA[2.2]~\cite{Bothmann:2019yzt} generator. In this set-up, NLO-accurate
matrix elements for up to one parton, and LO-accurate matrix elements for up
to three partons were calculated with the Comix~\cite{Gleisberg:2008fv} and
\OPENLOOPS~\cite{Buccioni:2019sur,Cascioli:2011va,Denner:2016kdg} libraries. They were matched
with the \SHERPA parton shower~\cite{Schumann:2007mg} using the \MEPSatNLO
prescription~\cite{Hoeche:2011fd,Hoeche:2012yf,Catani:2001cc,Hoeche:2009rj}
with a dynamic merging cut~\cite{Siegert:2016bre} of \qty{10}{\GeV}.
Photons were required to be isolated according to a smooth-cone isolation
criterion~\cite{Frixione:1998jh}. Samples were generated using the
\NNPDF[3.0nnlo] PDF set~\cite{Ball:2014uwa}, along with the dedicated set of tuned
parton-shower parameters developed by the \SHERPA authors.


\paragraph{Long description:}

Prompt diphoton production was simulated with the
\SHERPA[2.2]~\cite{Bothmann:2019yzt} parton shower Monte Carlo
generator. In this set-up, NLO and LO-accurate
matrix elements were calculated with the Comix~\cite{Gleisberg:2008fv} and
\OPENLOOPS~\cite{Buccioni:2019sur,Cascioli:2011va,Denner:2016kdg} libraries.
The default \SHERPA parton shower~\cite{Schumann:2007mg} based on
Catani--Seymour dipole factorisation and the cluster hadronisation model~\cite{Winter:2003tt}
were used. They employed the dedicated set of tuned parameters developed by the
\SHERPA authors for this generator version and the \NNPDF[3.0nnlo]
PDF set~\cite{Ball:2014uwa}.

The NLO matrix elements for a given jet multiplicity were matched to the parton
shower using a colour-exact variant of the MC@NLO algorithm~\cite{Hoeche:2011fd}.
Different jet multiplicities were then merged
into an inclusive sample using an improved CKKW matching
procedure~\cite{Catani:2001cc,Hoeche:2009rj} which was extended to NLO
accuracy using the \MEPSatNLO prescription~\cite{Hoeche:2012yf}.
The merging cut was set dynamically to a scale of \qty{20}{\GeV},
according to the prescription in Ref.~\cite{Siegert:2016bre}.

The renormalisation and factorisation scales for the diphoton core process
were set to the invariant mass of the photon pair, \(m_{\gamma\gamma}\).
The strong coupling constant was set to \(\alphas (m_Z)= 0.118\) and the QED coupling
constant was evaluated in the Thomson limit. Photons from the matrix elements were required to be central,
by being within the rapidity range \(|y_{\gamma}|<2.7\), and isolated according to a smooth-cone isolation
criterion~\cite{Frixione:1998jh} with \(\delta_0=0.1\), \(\epsilon_{\gamma}=0.1\) and \(n=2\).
Additionally, the photons were required to be separated by \(\Delta R(\gamma_1,\gamma_2) > 0.2\).


The effects of QCD scale uncertainties were evaluated~\cite{Bothmann:2016nao} using
seven-point variations of the factorisation and renormalisation scales in the matrix elements.
The scales were varied independently by factors of \(0.5\) and \(2\), avoiding variations in opposite directions.

PDF uncertainties for the nominal PDF set were
evaluated using the 100 variation replicas, as well as \(\pm 0.001\) shifts
of \alphas. Additionally, the results were cross-checked using the central values of the
\CT[14nnlo]~\cite{Dulat:2015mca} and \MMHT[nnlo]~\cite{Harland-Lang:2014zoa}
PDF sets.
%\textcolor{red}{ Several Monte Carlo slices were generated to ensure good statistics over the whole phase space. In each of them, the photon pair was required to fall within a \(m_{\gamma\gamma}\) range that depends on the slice avoiding any double-counting when the different slices are combined.}


\section[Sherpa (MEPS@LO)]{\SHERPA (\MEPSatLO)}
%\label{sec:gammajets-sherpa-lo}

\subsection*{Samples}
%\label{sec:gammajets-sherpa-lo-samples}

The descriptions below correspond to the samples in \cref{tab:gammajets-sherpa-lo}.

\begin{table}[!htbp]
  \caption{\(\gamma\)+jets and  \(\gamma\gamma\)+jets samples with \SHERPA LO\@.}%
  \label{tab:gammajets-sherpa-lo}
  \centering
  \begin{tabular}{l l}
    \toprule
    DSID range & Description \\
    \midrule
    361039--361062 & single photon \\
    303727--303742 & diphoton \\
    700442         & EWK \(\gamma jj\) \\
    \bottomrule
  \end{tabular}
\end{table}


\subsection[y+jets]{\(\gamma\)+jets}
%\label{sec:gammajets-sherpa-lo-singlephoton}

\paragraph{Description:}

Prompt single-photon production was simulated using the \SHERPA[2.1]~\cite{Bothmann:2019yzt}
generator. The tree-level matrix elements, generated for up to three
additional partons, were merged with the initial- and final-state parton showers using the
\MEPSatLO prescription~\cite{Hoeche:2009rj}. The \CT[10nlo] set of PDFs~\cite{Lai:2010vv} was
used to parameterise the proton structure in conjunction with the dedicated set of tuned
parton-shower parameters developed by the \SHERPA authors for this generator version. A
modified version of the cluster model~\cite{Winter:2003tt} was used
for the description of the fragmentation into hadrons. Photons from the matrix elements were
required to be isolated according to a smooth-cone hadronic isolation criterion~\cite{Frixione:1998jh}
with \(\delta_0=0.3\), \(\epsilon_{\gamma}=0.025\) and \(n=2\).


\subsection[yy+jets]{ \(\gamma\gamma\)+jets}
%\label{sec:gammajets-sherpa-lo-diphoton}

\paragraph{Description:}

Prompt diphoton production was simulated using the \SHERPA[2.1]~\cite{Bothmann:2019yzt}
generator. The tree-level matrix elements, generated for up to two
additional partons, were merged with the initial- and final-state parton showers using the
\MEPSatLO prescription~\cite{Hoeche:2009rj}. The \CT[10nlo] set of PDFs~\cite{Lai:2010vv} was
used to parameterise the proton structure in conjunction with the dedicated set of tuned
parton-shower parameters developed by the \SHERPA authors for this generator version. A
modified version of the cluster model~\cite{Winter:2003tt} was used
for the description of the fragmentation into hadrons. Photons from the matrix elements were
required to be isolated according to a smooth-cone hadronic isolation criterion~\cite{Frixione:1998jh}
with \(\delta_0=0.3\), \(\epsilon_{\gamma}=0.025\) and \(n=2\). Additionally, the photons were
required to be separated by \(\Delta R(\gamma_1,\gamma_2) > 0.2\).


\subsection[yjj]{ \(\gamma jj\)}

The descriptions below correspond to the samples in
\cref{tab:ewkyjets-sherpa}. The samples do not overlap with the QCD \(\gamma+\)jets samples.

\begin{table}[!htbp]
  \caption{Electroweak \(\gamma jj\) samples with \SHERPA.}%
  \label{tab:ewkyjets-sherpa}
  \centering
  \begin{tabular}{l l}
    \toprule
    DSID range & Description \\
    \midrule
    700442 & EWK \(\gamma jj\) \\
    \bottomrule
  \end{tabular}
\end{table}

\paragraph{Description:}

Electroweak production of the \(\gamma jj\) final state
was simulated with \SHERPA[2.2.11]~\cite{Bothmann:2019yzt} using
leading-order (LO) matrix elements with up to one additional parton emission.
The matrix elements were merged with the \SHERPA parton
shower~\cite{Schumann:2007mg} following the \MEPSatLO
prescription~\cite{Catani:2001cc} and using the set of tuned
parameters developed by the \SHERPA authors.
The \NNPDF[3.0nnlo] set of
PDFs~\cite{Ball:2014uwa} was employed. The samples were produced
using the VBF approximation, which avoids overlap with semileptonic
diboson topologies by requiring a \(t\)-channel colour-singlet exchange.
The starting conditions of the CS shower are set according to the
large-\(N_c\) amplitudes supplied by Comix~\cite{Buckley:2021gfw} to achieve
the correct VBF-appropriate radiation pattern.
Photons from the matrix elements were required to be isolated according to a smooth-cone 
hadronic isolation criterion~\cite{Frixione:1998jh} with 
\(\delta_0=0.1\), \(\epsilon_{\gamma}=0.1\) and \(n=2\).



\section[Pythia (LO)]{\PYTHIA (LO)}
%\label{sec:gammajets-pythia-lo}

The descriptions below correspond to the samples in \cref{tab:gammajets-pythia-lo}.

\begin{table}[!htbp]
  \caption{\(\gamma\)+jets and  \(\gamma\gamma\)+jets samples with \PYTHIA.}%
  \label{tab:gammajets-pythia-lo}
  \centering
  \begin{tabular}{l l}
    \toprule
    DSID range & Description \\
    \midrule
    423099--423112 & single photon \\
    344008, 302520--34, 364423 & diphoton \\
    \bottomrule
  \end{tabular}
\end{table}


\subsection[y+jets]{\(\gamma\)+jets}
%\label{sec:gammajets-pythia-lo-singlephoton}

\paragraph{Description:}

Prompt single-photon production was simulated using the \PYTHIA[8.186]~\cite{Sjostrand:2007gs} generator.
Events were generated using tree-level matrix elements for photon-plus-jet final
states as well as LO QCD dijet events, with the inclusion of initial-
and final-state parton showers. The fragmentation component was
modelled by final-state QED radiation arising from calculations of all
\(2\rightarrow 2\) QCD processes. The \NNPDF[2.3lo]~\cite{Ball:2012cx} PDF
set was used in the matrix element calculation, the parton shower, and
the simulation of the multi-parton interactions. The samples
include a simulation of the underlying event with parameters set
according to the A14 tune~\cite{ATL-PHYS-PUB-2014-021}. The Lund
string model~\cite{Andersson:1983ia,Sjostrand:1984ic} was used for the
description of the fragmentation into hadrons.


\subsection[yy+jets]{ \(\gamma\gamma\)+jets}
%\label{sec:gammajets-pythia-lo-diphoton}

\paragraph{Description:}

Prompt diphoton production was simulated using the
\PYTHIA[8.186]~\cite{Sjostrand:2007gs} generator. Events were
generated using tree-level matrix elements for diphoton final states,
with the inclusion of initial- and final-state parton showers.  The
fragmentation component was modelled by final-state QED radiation
arising from calculations of photon-plus-jet processes in dedicated
samples. The \NNPDF[2.3lo]~\cite{Ball:2012cx} PDF set was used in the
matrix element calculation, the parton shower, and in the simulation of the
multi-parton interactions. The samples include a simulation of the
underlying event with parameters set according to the A14
tune~\cite{ATL-PHYS-PUB-2014-021}. The Lund string
model~\cite{Andersson:1983ia,Sjostrand:1984ic} was used for the
description of the fragmentation into hadrons.
