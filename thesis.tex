% !TeX TXS-SCRIPT = openproj
% //Trigger = ?load-this-file
% files = [
%   "./chapters/chapter-1.tex",
%   "./chapters/chapter-2.tex",
%	"./chapters/chapter-3.tex",
%	"./chapters/chapter-4.tex",
%	"./chapters/chapter-5.tex",
% ]
% for (i in files) {
	%   files[i] = app.getAbsoluteFilePath(files[i]);
	% }
% for (i in files) {
	%   app.load(app.getAbsoluteFilePath(files[i]));
	% }
% TXS-SCRIPT-END


%% Before beginning to type your dissertation, download and READ
%% the Graduate School's formatting guide, which can be found at
%% http://grad.msu.edu/etd
%% and clicking Formatting Guide in the left hand column.
%% Also get the latest version of  msuphddissertation.cls and the template file
%% at http://www.math.msu.edu/~weil/MSU_Ph.D._Dissertation.zip
%% Send questions to weil@math.msu.edu

%%%%%%%%%%%%%%%%%%%%%%%%%%%%
%%%%%%%%  NOTE   %%%%%%%%%%%%%%
%% PREPARING A DISSERTATION WITH THIS CLASS FILE DOES NOT %%%
%% GUARANTEE THAT THE GRADUATE SCHOOL WILL APPROVE IT. %%%
%%%%%%%%%%%%%%%%%%%%%%%%%%%%%%%

%% To view a video presentation of this template, visit
%%  https://www.math.msu.edu/latex/dissertation/

\documentclass{msuphddissertation}
\usepackage{graphicx,float,wrapfig}
\usepackage{tabularx, amssymb}
\usepackage{amsmath}
\usepackage{mathtools}
\usepackage{subfig}
\usepackage{adjustbox}
\usepackage{pdfpages}
\usepackage{acronym}
\usepackage{booktabs}
\usepackage{longtable, tabu}
\usepackage{indentfirst}
\usepackage{wrapfig}
\usepackage[flushleft]{threeparttable}
\usepackage{hyperref}
\hypersetup{
colorlinks=true,
allcolors=black,
urlcolor=blue,
citecolor=blue
%hidelinks for plain black text
}
\usepackage[nottoc, numbib]{tocbibind}
\usepackage{xspace}
\usepackage{relsize}
\usepackage{appendix}
\usepackage{multicol, multirow}
\UseRawInputEncoding
\usepackage{lineno}
\usepackage{capt-of}
\linenumbers
\usepackage{url}
\urlstyle{same}
\usepackage[backend=bibtex,style=numeric-comp,bibencoding=utf8,natbib=true,sorting=none,backref,backrefstyle=three,firstinits=true,maxbibnames=5]{biblatex}
%\usepackage[backend=bibtex,style=ieee,bibencoding=utf8,natbib=true,sorting=none,backref]{biblatex}
\DeclareFieldFormat*{title}{\mkbibitalic{#1}}
\DeclareFieldFormat*{citetitle}{\mkbibitalic{#1}}
\DeclareFieldFormat{journaltitle}{#1\isdot}
\DeclareNameAlias{sortname}{family-given}
\renewbibmacro{in:}{}
\ExecuteBibliographyOptions{doi=false}
\DeclareFieldFormat{doilink}{%
\iffieldundef{doi}{#1}%
{\href{http://dx.doi.org/\thefield{doi}}{#1}}}
\DeclareBibliographyDriver{article}{%
  \usebibmacro{bibindex}%
  \usebibmacro{begentry}%
  \usebibmacro{author/translator+others}%
  \setunit{\labelnamepunct}\newblock
  \usebibmacro{title}%
  \newunit
  \printlist{language}%
  \newunit\newblock
  \usebibmacro{byauthor}%
  \newunit\newblock
  \usebibmacro{bytranslator+others}%
  \newunit\newblock
  \printfield{version}%
  \newunit\newblock
  \usebibmacro{in:}%
  \printtext[doilink]{%
  \usebibmacro{journal+issuetitle}%
  \newunit
  \usebibmacro{byeditor+others}%
  \newunit
  \usebibmacro{note+pages}%
  }%
  \newunit\newblock
  \iftoggle{bbx:isbn}
    {\printfield{issn}}
    {}%
  \newunit\newblock
  \usebibmacro{doi+eprint+url}%
  \newunit\newblock
  \usebibmacro{addendum+pubstate}%
  \setunit{\bibpagerefpunct}\newblock
  \usebibmacro{pageref}%
  \usebibmacro{finentry}}
\bibliography{bib/citations}
\bibliography{bib/ANA-EXOT-2022-44-INT1.bib}
\bibliography{bib/ATLAS.bib}
\bibliography{bib/ATLAS-useful.bib}
\bibliography{bib/CMS.bib}
\bibliography{bib/ConfNotes.bib}
\bibliography{bib/PubNotes.bib}
\usepackage{subfiles}
\usepackage{bm}
\usepackage{thesis-defs}
\setcounter{tocdepth}{5}
\setcounter{secnumdepth}{5}
\newcommand{\tocite}{\textcolor{red}{(citation)}\xspace}

%\titleformat{\chapter}[block]{\bfseries\fontsize{22pt}{22pt}\selectfont}{\chaptertitlename~\thechapter.}{10pt}{}

%\newcommand{\volume}{{\ooalign{\hfil$V$\hfil\cr\kern0.08em--\hfil\cr}}} %This command defines the "Volume Symbol," the V with a line through it. 

%% This is the first command that must appear in your thesis.
%% Insert packages you wish to use except setspace, subfig
%% geometry and pdflscape.
%% These packages are loaded automatically.
%% IMPORTANT: Load only those packages you know you will use.
%% Some packages can cause conflicts resulting in improper formatting.

\author{Hieu Le} %% Put your name in full as it is officially recognized by Michigan State University here.
\title{Search for $\lowercase{\ttbar}Z' \rightarrow \lowercase{\tttt}$ production in the multilepton final state in $\lowercase{pp}$ collisions at \lowercase{$\Ecom=$} 13 TeV with the ATLAS detector} %% Put the title of your dissertation here.

%% Go to http://grad.msu.edu/etd/docs/DegreeGrantingUnits.pdf
%% and find your GRADUATE DEGREE GRANTING UNIT/PROGRAM
%% and DEGREE.
\unit{Physics --- Doctor of Philosophy} %% Copy and paste these two items here
%% separated by a dash, created by typing ---.
%% ONLY ENTRIES FROM LIST MENTIONED ARE ACCEPTABLE!!

%% Put additional preamble items here.

\begin{document}
\maketitlepage %%This command will produce the title page of your thesis.

%% If you wish to include a "public abstract" (i.e.; in layman's terms), remove the "%"
%% in from of the command \begin{pub abstract} and remove the "%" in front of
%% \end{pub abstract} below. A public abstract isn't required, but might be useful
%% for some readers.
%\begin{pubabstract}
%%Type the text of your public abstract here. A public abstract is optional.
%\end{pubabstract}

\begin{abstract}
%This dissertation presents a search for a beyond-the-Standard-Model (BSM) top-philic heavy vector resonance based on a simplified top-philic color-singlet $Z'$($\rightarrow\ttbar$) produced in association with a top-quark pair (\ttZp), probing a $Z'$ mass range from 1 TeV to 3 TeV. The analysis targets the same-sign dilepton and multilepton final states of four-top-quark (\tttt) events and is highly motivated by the recent observation of excess in Standard Model (SM) \tttt production; the targeted search channels are sensitive to the \ttZp signal due to the distinctive \tttt signal signature with high jet multiplicity and the suppression of common SM background processes. The search is performed using the full Run 2 data set collected by the ATLAS detector at the Large Hadron Collider between 2015 and 2018, corresponding to an integrated luminosity of 140 fb$^{-1}$ of proton-proton collisions at a center-of-mass energy of $\sqrt{s} = 13$ TeV. Data-driven techniques for \ttW and charge misidentification background estimation are incorporated to improve background modeling at high jet multiplicity. No statistically significant deviation from SM predictions is observed. Observed exclusion limits at 95\% confidence level are set on the production cross section of $pp \rightarrow \ttZp$ multiplied by the branching ratio of $Z'\rightarrow\ttbar$ and range from 7.9 fb to 9.4 fb depending on the $Z'$ mass.

This dissertation presents a search for a new beyond-the-Standard-Model (BSM) particle at the Large Hadron Collider (LHC). Many BSM models predict a new heavy vector boson ($Z'$) that couples primarily to the top quark in both production and decay (top-philic). The search is performed in multilepton events consistent with four-top-quark (\tttt) production,
%The targeted signal for the search is the multilepton final state of four-top-quark (\tttt) events resulting from the production and decay of a $Z'$ particle in association with a top-quark pair (\ttZp). 
due to the distinctive signature of the multilepton final states and the its robustness against common background processes at the LHC. Analysis data was collected by the ATLAS detector from 2015 to 2018, using proton-proton collisions at the LHC at a center-of-mass energy of 13 TeV.
%Data-driven techniques for \ttW and charge misidentification background estimation are employed to improve modeling for background events with a large number of jets.
No statistically significant deviation from Standard Model predictions is observed. Exclusion limits are set on the production cross section of the targeted top-philic particle in the mass range between 1 TeV and 3 TeV.

\end{abstract}

%% If you wish to have a copyright page, remove the "%"
%% in front of \begin{copyrt}
%% and remove the "%" in front of \end{copyrt}.
%% An acceptable form of a copyright page
%% will be generated automatically.
%% TO INCLUDE A COPYRIGHT, YOU MUST REGISTER
%% IT. See the Formatting Guide for instructions.
%\begin{copyrt}
%\end{copyrt}


%% If you wish to have a dedication, remove the "%" in front of
%% \begin{dedication} and remove the "%" in front of
%% \end{dedication} below.
%% A dedication must be single-spaced and
%% centered on the page. Both will be done automatically.

%\begin{dedication}
%I dedicate this work to the Opossum and his noble pursuit of snacks.
%\end{dedication}

%% If you wish to have an acknowledgment, remove the "%" in front of  \begin{acknowledgment}
%% and remove the "%" in front of  \end{acknowledgment} below.
\begin{acknowledgment}

First and foremost, I am deeply grateful for my dissertation advisor and P.I, Professor Reinhard Schwienhorst, for his support, guidance and tolerance as part of my role in ATLAS and my doctoral program at Michigan State. Reinhard is the primary driving force in many exciting opportunities that I've had the chance to experience, and he also provides much-appreciated support both in knowledge and wisdom in times of need. I am incredibly thankful that Reinhard is one of the people that plays a part in who I am today.

I would like to express sincere gratitude to one of our postdocs in the MSU ATLAS group, Binbin Dong, who I closely worked with within ATLAS. Binbin is a massive source of support for physics, technical and ATLAS-specific knowledge that played a pivotal role during my training with ATLAS, during the analysis in this dissertation, and during my time at CERN. I would have never been able to find my way through without her help.

I am also extremely thankful for the MSU ATLAS group, in particular Professors Wade Fisher and Daniel Hayden, for their guidance and feedback on my professional and personal endeavors which helped immensely in my development both scientifically and socially. I thank Rongqian Qian and Jason Gombas, my fellow advisees that offered great ideas, knowledge and friendship. I would like to thank Julia Hinds, Stergios Kazakos and Pratik Kafle for their support and companionship during my time at CERN. I also thank former and presents members of our group that I've had the pleasure to work with: Joey Huston, José Gabriel Reyes Rivera, Cecilia Imthurn, Xinfei Huang, Ahmed Tarek, Kyle Fielman, Robert Les and Trisha Farooque. It was a wonderful experience being part of the MSU ATLAS group and I hope our group continues to grow, even if it makes scheduling weekly meetings for everyone that much harder.

I would like to express my gratitude to my dissertation committee members, Professors Reinhard Schwienhorst, Johannes Pollanen, Wade Fisher, Remco Zegers and Yuying Xie, for their guidance, patience and commitment to my growth and success as a researcher and a person.

It has been a pleasure to work with the many outstanding people in ATLAS, especially the BSM multi-top analysis team. I would like to thank Philipp Gadow, Krisztian Peters, Fr\'{e}d\'{e}ric D\'{e}liot and Neelam Kumari for their dedication and commitment to fostering a successful and fruitful collaboration. I also thank Meng-Ju Tsai, Hui-Chi Lin, Thomas Nommensen, Jianming Qian, Quake Qin, Tomke Schr\"{o}er, Xilin Wang, Helena Gomez and Daniela Paredes for their tireless efforts in the analysis. I am truly glad to have had the chance to work with all of you.

Special thanks to my fellow graduate students that I have had the chance to befriend during my doctoral journey: Daniel Lay, Grayson Perez, Jordan Purcell, Eric Flynn, Isabella Molina, Mo Hassan, Cavan Maher and Hannah Berg. You all taught me a lot more than I could ever imagine and helped me more than I could ever asked for, and I look forward to see where we go from here.

Finally, I would like to thank my family, to whom this dissertation is dedicated: my spouse Allen Sechrist, for encouraging me tirelessly everyday and always being there for me even when I can't be there for myself; my cat Eddie, for being the best cat anyone could ask for; my brother Hien Le, my dad Bac Le, and my mom Thuy Cao, for their endless love and support. Thank you for being the reason that I am where I am today.
\end{acknowledgment}

%% If you wish to have a preface, remove the "%" in front of \begin{preface}
%% and remove the "%" in front of \end{preface} below. The formatting of
%% a preface isn't specified, but it is included in the TOC.
%\begin{preface}
%This is my preface. 
%remarks
%remarks
%remarks
%\end{preface}

\TOC %% This command produces the Table of Contents. DO NOT REMOVE IT!


%% If your document contains tables, remove the "%" in front of
%%  the following line.
%%List of Tables is no longer required by MSU formatting requirements
%%but may be optionally included
\LOT

%% If your document contains figures, remove the "%" in front of
%% the following line.
%%List of Figures is no longer required by MSU formatting requirements
%%but may be optionally included
\LOF

%%%% LIST OF SYMBOLS OR LIST OF ABBREVIATIONS %%%%
%% If you wish to have a list of symbols or a list of abbreviations,
%% it should be here. For a list of symbols remove the "%" in front of
%% \begin{symbols} and remove the "%" in front of \end{symbols} below.
%\begin{symbols}
%% Type your list using a list environment here.
%\end{symbols}
%% Similarly for a list of abbreviations remove the "%" in front of
%% \begin{abbrev} and remove the "%" in front of \end{abbrev} below.
% Type your list using a list environment here.
\subfile{abbreviations}
%% The list will be included in the TOC as
%% KEY TO SYMBOLS or KEY TO ABBREVIATIONS
%%%%%%%%%%

\newpage
\pagenumbering{arabic}
\begin{doublespace}


%This block surrounds bosdy of dissertation and makes chapter titles better looking.
%It must only be used on the body of the dissertation because it will mess up TOC and BIBLIOGRAPHY titles
{
\FormatChapterTitles  %This command makes Chapter titles look nice and meet formatting requirements. It is defined in the .cls document

%###############################################################################################
%% Put the body of your dissertation here.
%% DO NOT include the bibliography or any appendices.
%% These topics will be discussed later.

%%copy this block to create more chapters
%---------------------------------------------------------------------------------
%---------------------------------------------------------------------------------

\chapter{Introduction}
\label{chap:intro}
\subfile{chapters/chapter-1}

\chapter{Theoretical Overview}
\label{chap:theory}
\subfile{chapters/chapter-2}

\chapter{LHC \& ATLAS Experiment}
\label{chap:LHCATLAS}
\subfile{chapters/chapter-3}

\chapter{Particle Reconstruction \& Identification}
\label{chap:reco}
\subfile{chapters/chapter-5}

\chapter{Data \& Simulated Samples}
\label{chap:samples}
\subfile{chapters/chapter-4}

\chapter{Analysis Strategy}
\label{chap:anal}
\subfile{chapters/chapter-7}

\chapter{Systematic Uncertainties}
\label{chap:sys}
\subfile{chapters/chapter-8}

\chapter{Results}
\label{chap:results}
\subfile{chapters/chapter-9}

\chapter{Summary}
\label{chap:summary}
\subfile{chapters/chapter-10}

}%end formatchapters block

%Here ends the body of your dissertation.
%###############################################################################################

%\begin{figure}[!htbp]
%\begin{center}
%\subcaptionbox{}{
%\includegraphics[width=0.35\linewidth]{figures/signals/ttHA/m400_w5_event_Inclusive_HT_normalized.pdf}}
%\subcaptionbox{}{
%\includegraphics[width=0.35\linewidth]{figures/signals/ttHA/m1000_w30_event_Inclusive_HT_normalized.pdf}}
%\subcaptionbox{}{
%\includegraphics[width=0.35\linewidth]{figures/signals/ttHA/m400_w5_parton_Inclusive_InvM_ttbar_1_normalize.pdf}}
%\subcaptionbox{}{
%\includegraphics[width=0.35\linewidth]{figures/signals/ttHA/m1000_w30_parton_Inclusive_InvM_ttbar_1_normalize.pdf}}
%\caption{\label{fig:setup_change} Kinematics comparison before and after aligning with the SM $t\Bar{t}t\Bar{t}$ sample. (a),(b): $H_T$, (c),(d): $t\Bar{t}$ invariant mass. The change of kinematics is mostly due to the change of dynamical scale choice from $H_T$ to $H_T/4$.}
%\end{center}
%\end{figure}


%%%%%% LANDSCAPE PAGES  %%%%%%
%% To produce graphics or tables in landscape mode,
%% begin by removing the "%" in the next two lines.
%\begin{landscape}
%\thispagestyle{empty}
%% The contents of the page can be centered using the center environment
%% or the \centering command. Insert either a table with the tabular environment
%% or input a graphics file.
%% Use \captionof{table}{caption_text} (or figure in place of table)
%% to create the caption for a short table or a figure.
%% Insert a long table in a table environment. It disables double spacing
%% which is permitted for long tables. Finally remove the "%" from the next line.
%\end{landacape}


%%%%%%  Bibliography %%%%%
%% A bibliography is required. By default it is called, "Bibliography"
%% You may use �LITERATURE CITED�, �WORKS CITED� or �REFERENCES�
%% instead of �BIBLIOGRAPHY� if that is the convention in your discipline.
%% To do so, copy and paste your choice into the empty argument
%% of the following command and remove the "%".
%\renewcommand{\bibname}{References}
%% The bibliography may be made using BibTeX.
%% To do so the necessary commands must be entered in the
%%uncomment the following two lines.
%\bibliographystyle{ieeetr}

%\renewcommand*{\UrlFont}{\rmfamily}
\printbibliography[heading=bibintoc,title={References}]


%% If the Bibliography is made from scratch,
%% remove the "%" in front of \begin{thebibliography}{'''}
%% replacing the ''' with the appropriate entry and
%% remove the "%" in front of \end{thebibliography} below.
% \begin{thebibliography}{'''}
%%  Enter the bibliography here.
% \end{thebibliography}
%% In either case, the bibliography is automatically entered
%% in the Table of Contents.


%%%%%%%    APPENDICES    %%%%%%%%%%
%% To start your first appendix, which will be labeled
%% as Appendix A, just type \chapter{<appendix 1 name>}
%% and enter the text of the appendix as you would a chapter,
%% with one exception. If you use any subdivisions, such as
%% section, subsection etc., use the starred version; that is,
%% \section*{section name}.  Such subdivisions are not to be
%% listed in the Table of Contents.

%%%%%%% A NOTE ABOUT APPENDICES %%%%%%%%%
%% Some appendices may be single-spaced such as survey examples
%% or letters. See the Graduate School's formatting guide for details.
%% To single space an appendix first remove the "%" from
%% the following two lines.
%\end{doublespace}
%\chapter{<appendix name>}
%% Insert the name of the appendix.
%% Insert the text of the appendix.
%% Remove the "%" from the following line.
%\begin{doublespace}
%% Any text entered now will be double spaced.

{
\FormatAppTitles %pretty chapter headings for Appendices. It is defined in the .cls document


\appendix

%Copy this block to add additional Appendices. It contains the requisite spacing commands. 
%---------------------------------------------------------------------------------
%---------------------------------------------------------------------------------
%\chapter{\qquad Statistical analysis}
%\label{appendix:stat}
%\subfile{chapters/appendix-A}


}%End format appendices block

%These commands close out the doublespace area and end the document.
\end{doublespace}
\end{document}

%%%%%% FINAL COMMENTS %%%%
%% Before submitting your dissertation to the Graduate School
%% Make sure there isn't any text in the right margins. To do
%% in the .log file look for error messages beginning with,
%% "Overfull \hbox ".

%% Once your document has been filed with the Graduate School,
%% if you wish to produce a single spaced version of your document,
%% find and remove the commands \begin{doublespace}
%% and \end{doublespace} above.
